\PassOptionsToPackage{unicode=true}{hyperref} % options for packages loaded elsewhere
\PassOptionsToPackage{hyphens}{url}
%
\documentclass[]{article}
\usepackage{lmodern}
\usepackage{amssymb,amsmath}
\usepackage{ifxetex,ifluatex}
\usepackage{fixltx2e} % provides \textsubscript
\ifnum 0\ifxetex 1\fi\ifluatex 1\fi=0 % if pdftex
  \usepackage[T1]{fontenc}
  \usepackage[utf8]{inputenc}
  \usepackage{textcomp} % provides euro and other symbols
\else % if luatex or xelatex
  \usepackage{unicode-math}
  \defaultfontfeatures{Ligatures=TeX,Scale=MatchLowercase}
\fi
% use upquote if available, for straight quotes in verbatim environments
\IfFileExists{upquote.sty}{\usepackage{upquote}}{}
% use microtype if available
\IfFileExists{microtype.sty}{%
\usepackage[]{microtype}
\UseMicrotypeSet[protrusion]{basicmath} % disable protrusion for tt fonts
}{}
\IfFileExists{parskip.sty}{%
\usepackage{parskip}
}{% else
\setlength{\parindent}{0pt}
\setlength{\parskip}{6pt plus 2pt minus 1pt}
}
\usepackage{hyperref}
\hypersetup{
            pdftitle={TP\_0\_Analyse\_données},
            pdfauthor={EL\_Hadrami},
            pdfborder={0 0 0},
            breaklinks=true}
\urlstyle{same}  % don't use monospace font for urls
\usepackage[margin=1in]{geometry}
\usepackage{color}
\usepackage{fancyvrb}
\newcommand{\VerbBar}{|}
\newcommand{\VERB}{\Verb[commandchars=\\\{\}]}
\DefineVerbatimEnvironment{Highlighting}{Verbatim}{commandchars=\\\{\}}
% Add ',fontsize=\small' for more characters per line
\usepackage{framed}
\definecolor{shadecolor}{RGB}{248,248,248}
\newenvironment{Shaded}{\begin{snugshade}}{\end{snugshade}}
\newcommand{\AlertTok}[1]{\textcolor[rgb]{0.94,0.16,0.16}{#1}}
\newcommand{\AnnotationTok}[1]{\textcolor[rgb]{0.56,0.35,0.01}{\textbf{\textit{#1}}}}
\newcommand{\AttributeTok}[1]{\textcolor[rgb]{0.77,0.63,0.00}{#1}}
\newcommand{\BaseNTok}[1]{\textcolor[rgb]{0.00,0.00,0.81}{#1}}
\newcommand{\BuiltInTok}[1]{#1}
\newcommand{\CharTok}[1]{\textcolor[rgb]{0.31,0.60,0.02}{#1}}
\newcommand{\CommentTok}[1]{\textcolor[rgb]{0.56,0.35,0.01}{\textit{#1}}}
\newcommand{\CommentVarTok}[1]{\textcolor[rgb]{0.56,0.35,0.01}{\textbf{\textit{#1}}}}
\newcommand{\ConstantTok}[1]{\textcolor[rgb]{0.00,0.00,0.00}{#1}}
\newcommand{\ControlFlowTok}[1]{\textcolor[rgb]{0.13,0.29,0.53}{\textbf{#1}}}
\newcommand{\DataTypeTok}[1]{\textcolor[rgb]{0.13,0.29,0.53}{#1}}
\newcommand{\DecValTok}[1]{\textcolor[rgb]{0.00,0.00,0.81}{#1}}
\newcommand{\DocumentationTok}[1]{\textcolor[rgb]{0.56,0.35,0.01}{\textbf{\textit{#1}}}}
\newcommand{\ErrorTok}[1]{\textcolor[rgb]{0.64,0.00,0.00}{\textbf{#1}}}
\newcommand{\ExtensionTok}[1]{#1}
\newcommand{\FloatTok}[1]{\textcolor[rgb]{0.00,0.00,0.81}{#1}}
\newcommand{\FunctionTok}[1]{\textcolor[rgb]{0.00,0.00,0.00}{#1}}
\newcommand{\ImportTok}[1]{#1}
\newcommand{\InformationTok}[1]{\textcolor[rgb]{0.56,0.35,0.01}{\textbf{\textit{#1}}}}
\newcommand{\KeywordTok}[1]{\textcolor[rgb]{0.13,0.29,0.53}{\textbf{#1}}}
\newcommand{\NormalTok}[1]{#1}
\newcommand{\OperatorTok}[1]{\textcolor[rgb]{0.81,0.36,0.00}{\textbf{#1}}}
\newcommand{\OtherTok}[1]{\textcolor[rgb]{0.56,0.35,0.01}{#1}}
\newcommand{\PreprocessorTok}[1]{\textcolor[rgb]{0.56,0.35,0.01}{\textit{#1}}}
\newcommand{\RegionMarkerTok}[1]{#1}
\newcommand{\SpecialCharTok}[1]{\textcolor[rgb]{0.00,0.00,0.00}{#1}}
\newcommand{\SpecialStringTok}[1]{\textcolor[rgb]{0.31,0.60,0.02}{#1}}
\newcommand{\StringTok}[1]{\textcolor[rgb]{0.31,0.60,0.02}{#1}}
\newcommand{\VariableTok}[1]{\textcolor[rgb]{0.00,0.00,0.00}{#1}}
\newcommand{\VerbatimStringTok}[1]{\textcolor[rgb]{0.31,0.60,0.02}{#1}}
\newcommand{\WarningTok}[1]{\textcolor[rgb]{0.56,0.35,0.01}{\textbf{\textit{#1}}}}
\usepackage{graphicx,grffile}
\makeatletter
\def\maxwidth{\ifdim\Gin@nat@width>\linewidth\linewidth\else\Gin@nat@width\fi}
\def\maxheight{\ifdim\Gin@nat@height>\textheight\textheight\else\Gin@nat@height\fi}
\makeatother
% Scale images if necessary, so that they will not overflow the page
% margins by default, and it is still possible to overwrite the defaults
% using explicit options in \includegraphics[width, height, ...]{}
\setkeys{Gin}{width=\maxwidth,height=\maxheight,keepaspectratio}
\setlength{\emergencystretch}{3em}  % prevent overfull lines
\providecommand{\tightlist}{%
  \setlength{\itemsep}{0pt}\setlength{\parskip}{0pt}}
\setcounter{secnumdepth}{0}
% Redefines (sub)paragraphs to behave more like sections
\ifx\paragraph\undefined\else
\let\oldparagraph\paragraph
\renewcommand{\paragraph}[1]{\oldparagraph{#1}\mbox{}}
\fi
\ifx\subparagraph\undefined\else
\let\oldsubparagraph\subparagraph
\renewcommand{\subparagraph}[1]{\oldsubparagraph{#1}\mbox{}}
\fi

% set default figure placement to htbp
\makeatletter
\def\fps@figure{htbp}
\makeatother


\title{TP\_0\_Analyse\_données}
\author{EL\_Hadrami}
\date{16/10/2020}

\begin{document}
\maketitle

\(\underline{\text{Exercice 1}}\)

\begin{Shaded}
\begin{Highlighting}[]
\CommentTok{# order c'est pour les indices}
\CommentTok{# Creations de trois vecteurs}
\NormalTok{x <-}\StringTok{ }\KeywordTok{c}\NormalTok{(}\DecValTok{1}\NormalTok{, }\DecValTok{3}\NormalTok{, }\DecValTok{5}\NormalTok{, }\DecValTok{7}\NormalTok{, }\DecValTok{9}\NormalTok{)}
\NormalTok{y <-}\StringTok{ }\KeywordTok{c}\NormalTok{(}\DecValTok{2}\NormalTok{, }\DecValTok{3}\NormalTok{, }\DecValTok{5}\NormalTok{, }\DecValTok{7}\NormalTok{, }\DecValTok{11}\NormalTok{, }\DecValTok{13}\NormalTok{) }
\NormalTok{z <-}\StringTok{ }\KeywordTok{c}\NormalTok{(}\DecValTok{9}\NormalTok{, }\DecValTok{3}\NormalTok{, }\DecValTok{2}\NormalTok{, }\DecValTok{5}\NormalTok{, }\DecValTok{9}\NormalTok{, }\DecValTok{2}\NormalTok{, }\DecValTok{3}\NormalTok{, }\DecValTok{9}\NormalTok{,}\DecValTok{1}\NormalTok{)}
\NormalTok{x }\OperatorTok{+}\StringTok{ }\DecValTok{2} 
\end{Highlighting}
\end{Shaded}

\begin{verbatim}
## [1]  3  5  7  9 11
\end{verbatim}

\begin{Shaded}
\begin{Highlighting}[]
\NormalTok{y }\OperatorTok{*}\StringTok{ }\DecValTok{3} 
\end{Highlighting}
\end{Shaded}

\begin{verbatim}
## [1]  6  9 15 21 33 39
\end{verbatim}

\begin{Shaded}
\begin{Highlighting}[]
\KeywordTok{length}\NormalTok{(x) }
\end{Highlighting}
\end{Shaded}

\begin{verbatim}
## [1] 5
\end{verbatim}

\begin{Shaded}
\begin{Highlighting}[]
\NormalTok{x }\OperatorTok{+}\StringTok{ }\NormalTok{y  }\CommentTok{# erreur}
\end{Highlighting}
\end{Shaded}

\begin{verbatim}
## Warning in x + y: la taille d'un objet plus long n'est pas multiple de la taille
## d'un objet plus court
\end{verbatim}

\begin{verbatim}
## [1]  3  6 10 14 20 14
\end{verbatim}

\begin{Shaded}
\begin{Highlighting}[]
\KeywordTok{sum}\NormalTok{(x }\OperatorTok{>}\StringTok{ }\DecValTok{5}\NormalTok{)}
\end{Highlighting}
\end{Shaded}

\begin{verbatim}
## [1] 2
\end{verbatim}

\begin{Shaded}
\begin{Highlighting}[]
\KeywordTok{sum}\NormalTok{(x[x }\OperatorTok{>}\StringTok{ }\DecValTok{5}\NormalTok{])}
\end{Highlighting}
\end{Shaded}

\begin{verbatim}
## [1] 16
\end{verbatim}

\begin{Shaded}
\begin{Highlighting}[]
\KeywordTok{sum}\NormalTok{(x }\OperatorTok{>}\StringTok{ }\DecValTok{5} \OperatorTok{|}\StringTok{ }\NormalTok{x }\OperatorTok{<}\StringTok{ }\DecValTok{3}\NormalTok{) }
\end{Highlighting}
\end{Shaded}

\begin{verbatim}
## [1] 3
\end{verbatim}

\begin{Shaded}
\begin{Highlighting}[]
\NormalTok{y[}\DecValTok{3}\NormalTok{] }
\end{Highlighting}
\end{Shaded}

\begin{verbatim}
## [1] 5
\end{verbatim}

\begin{Shaded}
\begin{Highlighting}[]
\NormalTok{y[}\OperatorTok{-}\DecValTok{3}\NormalTok{]}
\end{Highlighting}
\end{Shaded}

\begin{verbatim}
## [1]  2  3  7 11 13
\end{verbatim}

\begin{Shaded}
\begin{Highlighting}[]
\NormalTok{y[x] ; (y }\OperatorTok{>}\StringTok{ }\DecValTok{7}\NormalTok{) ; y[y }\OperatorTok{>}\StringTok{ }\DecValTok{7}\NormalTok{] ; }\KeywordTok{sort}\NormalTok{(z) ; }\KeywordTok{sort}\NormalTok{(z, }\DataTypeTok{dec =} \OtherTok{TRUE}\NormalTok{) ; }\KeywordTok{rev}\NormalTok{(z) ;}
\end{Highlighting}
\end{Shaded}

\begin{verbatim}
## [1]  2  5 11 NA NA
\end{verbatim}

\begin{verbatim}
## [1] FALSE FALSE FALSE FALSE  TRUE  TRUE
\end{verbatim}

\begin{verbatim}
## [1] 11 13
\end{verbatim}

\begin{verbatim}
## [1] 1 2 2 3 3 5 9 9 9
\end{verbatim}

\begin{verbatim}
## [1] 9 9 9 5 3 3 2 2 1
\end{verbatim}

\begin{verbatim}
## [1] 1 9 3 2 9 5 2 3 9
\end{verbatim}

\begin{Shaded}
\begin{Highlighting}[]
\KeywordTok{order}\NormalTok{(z) ; }\KeywordTok{unique}\NormalTok{(z) ; }\KeywordTok{duplicated}\NormalTok{(z) ; }\KeywordTok{table}\NormalTok{(z) ; }\KeywordTok{rep}\NormalTok{(z, }\DecValTok{3}\NormalTok{)}
\end{Highlighting}
\end{Shaded}

\begin{verbatim}
## [1] 9 3 6 2 7 4 1 5 8
\end{verbatim}

\begin{verbatim}
## [1] 9 3 2 5 1
\end{verbatim}

\begin{verbatim}
## [1] FALSE FALSE FALSE FALSE  TRUE  TRUE  TRUE  TRUE FALSE
\end{verbatim}

\begin{verbatim}
## z
## 1 2 3 5 9 
## 1 2 2 1 3
\end{verbatim}

\begin{verbatim}
##  [1] 9 3 2 5 9 2 3 9 1 9 3 2 5 9 2 3 9 1 9 3 2 5 9 2 3 9 1
\end{verbatim}

\(\underline{\text{Exercice 2}}\)

\begin{Shaded}
\begin{Highlighting}[]
\NormalTok{diag1 <-}\StringTok{ }\KeywordTok{diag}\NormalTok{(}\DecValTok{1}\NormalTok{,}\DataTypeTok{nrow=}\DecValTok{9}\NormalTok{,}\DataTypeTok{ncol=}\DecValTok{9}\NormalTok{) }\CommentTok{# creation d'une matrice diagonale}
\NormalTok{x1 <-}\StringTok{ }\KeywordTok{rep}\NormalTok{(}\DataTypeTok{x=}\DecValTok{1}\NormalTok{,}\DataTypeTok{times=}\DecValTok{9}\NormalTok{)}
\NormalTok{matrix1 <-}\StringTok{ }\KeywordTok{matrix}\NormalTok{(x1,}\DataTypeTok{nrow=}\DecValTok{9}\NormalTok{,}\DataTypeTok{ncol=}\DecValTok{9}\NormalTok{)}
\NormalTok{matdiag0 <-}\StringTok{ }\NormalTok{matrix1 }\OperatorTok{-}\StringTok{ }\NormalTok{diag1}
\end{Highlighting}
\end{Shaded}

\(\underline{\text{Exercice 3}}\)

\begin{Shaded}
\begin{Highlighting}[]
\CommentTok{# Creation de deux vecteurs}
\NormalTok{vect1 <-}\StringTok{ }\KeywordTok{c}\NormalTok{(}\DecValTok{1}\OperatorTok{:}\DecValTok{10}\NormalTok{)}
\NormalTok{vect2 <-}\StringTok{ }\KeywordTok{c}\NormalTok{(}\DecValTok{11}\OperatorTok{:}\DecValTok{20}\NormalTok{)}
\NormalTok{vectconcat <-}\StringTok{ }\KeywordTok{c}\NormalTok{(}\DecValTok{1}\NormalTok{,vect1[}\DecValTok{2}\NormalTok{],vect2,vect1[}\KeywordTok{c}\NormalTok{(}\DecValTok{3}\OperatorTok{:}\DecValTok{10}\NormalTok{)])}
\end{Highlighting}
\end{Shaded}

\(\underline{\text{Exercice 4}}\)

\begin{Shaded}
\begin{Highlighting}[]
\NormalTok{x <-}\StringTok{ }\KeywordTok{c}\NormalTok{ (}\FloatTok{4.12}\NormalTok{, }\FloatTok{1.84}\NormalTok{, }\FloatTok{4.28}\NormalTok{, }\FloatTok{4.23}\NormalTok{, }\FloatTok{1.74}\NormalTok{, }\FloatTok{2.06}\NormalTok{, }\FloatTok{3.37}\NormalTok{, }\FloatTok{3.83}\NormalTok{, }\FloatTok{5.15}\NormalTok{, }\FloatTok{3.76}\NormalTok{, }\FloatTok{3.23}\NormalTok{, }\FloatTok{4.87}\NormalTok{,}
\FloatTok{5.96}\NormalTok{, }\FloatTok{2.29}\NormalTok{, }\FloatTok{4.58}\NormalTok{)}
\NormalTok{x_extract1 <-}\StringTok{ }\NormalTok{x[(}\DecValTok{4}\OperatorTok{:}\DecValTok{15}\NormalTok{)] }\CommentTok{# extraction}
\NormalTok{x_extract2 <-}\StringTok{ }\NormalTok{x[(}\DecValTok{2}\OperatorTok{:}\DecValTok{14}\NormalTok{)]}
\NormalTok{x_extract3 <-}\StringTok{ }\NormalTok{x[x }\OperatorTok{>}\StringTok{ }\FloatTok{2.57} \OperatorTok{&}\StringTok{ }\NormalTok{x }\OperatorTok{<}\StringTok{ }\FloatTok{3.48}\NormalTok{]}
\NormalTok{x_extract4 <-}\StringTok{ }\NormalTok{x[x }\OperatorTok{>}\StringTok{ }\FloatTok{4.07}  \OperatorTok{||}\StringTok{ }\NormalTok{x }\OperatorTok{<}\StringTok{ }\FloatTok{1.48}\NormalTok{]}
\NormalTok{indice_min <-}\StringTok{ }\KeywordTok{which.min}\NormalTok{(x)}
\end{Highlighting}
\end{Shaded}

\(\underline{\text{Exercice 5}}\)

\begin{Shaded}
\begin{Highlighting}[]
\NormalTok{row1A <-}\StringTok{ }\KeywordTok{c}\NormalTok{(}\OperatorTok{-}\DecValTok{2}\NormalTok{,}\DecValTok{1}\NormalTok{,}\OperatorTok{-}\DecValTok{3}\NormalTok{,}\OperatorTok{-}\DecValTok{2}\NormalTok{)}
\NormalTok{row2B <-}\StringTok{ }\KeywordTok{c}\NormalTok{(}\DecValTok{1}\NormalTok{,}\DecValTok{2}\NormalTok{,}\DecValTok{1}\NormalTok{,}\OperatorTok{-}\DecValTok{1}\NormalTok{)}
\NormalTok{row3C <-}\StringTok{ }\KeywordTok{c}\NormalTok{(}\OperatorTok{-}\DecValTok{2}\NormalTok{,}\DecValTok{1}\NormalTok{,}\DecValTok{1}\NormalTok{,}\OperatorTok{-}\DecValTok{1}\NormalTok{)}
\NormalTok{row4D <-}\StringTok{ }\KeywordTok{c}\NormalTok{(}\OperatorTok{-}\DecValTok{1}\NormalTok{,}\OperatorTok{-}\DecValTok{3}\NormalTok{,}\DecValTok{1}\NormalTok{,}\DecValTok{1}\NormalTok{)}
\NormalTok{A <-}\StringTok{ }\KeywordTok{matrix}\NormalTok{(}\KeywordTok{c}\NormalTok{(row1A,row2B,row3C,row4D),}\DataTypeTok{nrow=}\DecValTok{4}\NormalTok{,}\DataTypeTok{ncol=}\DecValTok{4}\NormalTok{,}\DataTypeTok{byrow =} \OtherTok{TRUE}\NormalTok{)}
\NormalTok{rowA <-}\StringTok{ }\KeywordTok{c}\NormalTok{(}\DecValTok{2}\NormalTok{,}\OperatorTok{-}\DecValTok{1}\NormalTok{,}\DecValTok{3}\NormalTok{,}\OperatorTok{-}\DecValTok{4}\NormalTok{)}
\NormalTok{rowB <-}\StringTok{ }\KeywordTok{c}\NormalTok{(}\DecValTok{2}\NormalTok{,}\OperatorTok{-}\DecValTok{2}\NormalTok{,}\DecValTok{4}\NormalTok{,}\OperatorTok{-}\DecValTok{5}\NormalTok{)}
\NormalTok{rowC <-}\StringTok{ }\KeywordTok{c}\NormalTok{(}\OperatorTok{-}\DecValTok{2}\NormalTok{,}\DecValTok{1}\NormalTok{,}\DecValTok{3}\NormalTok{,}\OperatorTok{-}\DecValTok{1}\NormalTok{)}
\NormalTok{rowD <-}\StringTok{ }\KeywordTok{c}\NormalTok{(}\OperatorTok{-}\DecValTok{1}\NormalTok{,}\OperatorTok{-}\DecValTok{3}\NormalTok{,}\DecValTok{1}\NormalTok{,}\OperatorTok{-}\DecValTok{1}\NormalTok{)}
\NormalTok{B <-}\StringTok{ }\KeywordTok{matrix}\NormalTok{(}\KeywordTok{c}\NormalTok{(rowA,rowB,rowC,rowD),}\DataTypeTok{ncol =} \DecValTok{4}\NormalTok{,}\DataTypeTok{nrow =} \DecValTok{4}\NormalTok{,}\DataTypeTok{byrow =}\NormalTok{ T)}
\CommentTok{# Montrons que  A et B sont inversible}
\KeywordTok{det}\NormalTok{(A)}
\end{Highlighting}
\end{Shaded}

\begin{verbatim}
## [1] 25
\end{verbatim}

\begin{Shaded}
\begin{Highlighting}[]
\KeywordTok{det}\NormalTok{(B)}
\end{Highlighting}
\end{Shaded}

\begin{verbatim}
## [1] 4
\end{verbatim}

\begin{Shaded}
\begin{Highlighting}[]
\CommentTok{# det(A) et det(B) ne sont pas nuls donc A et B sont inversible}
\CommentTok{# L'inverse des matrices}
\KeywordTok{solve}\NormalTok{(A)}
\end{Highlighting}
\end{Shaded}

\begin{verbatim}
##       [,1]  [,2]  [,3] [,4]
## [1,]  0.08  0.48 -0.44  0.2
## [2,] -0.24 -0.44  0.32 -0.6
## [3,] -0.12  0.28  0.16  0.2
## [4,] -0.52 -1.12  0.36 -0.8
\end{verbatim}

\begin{Shaded}
\begin{Highlighting}[]
\KeywordTok{solve}\NormalTok{(B)}
\end{Highlighting}
\end{Shaded}

\begin{verbatim}
##        [,1]  [,2]  [,3] [,4]
## [1,]  -7.50  6.50 -0.50 -2.0
## [2,]   3.25 -2.75  0.25  0.5
## [3,] -10.25  8.75 -0.25 -2.5
## [4,] -12.50 10.50 -0.50 -3.0
\end{verbatim}

\begin{Shaded}
\begin{Highlighting}[]
\CommentTok{#2.}
\KeywordTok{det}\NormalTok{(}\KeywordTok{t}\NormalTok{(A))}
\end{Highlighting}
\end{Shaded}

\begin{verbatim}
## [1] 25
\end{verbatim}

\begin{Shaded}
\begin{Highlighting}[]
\KeywordTok{det}\NormalTok{(}\KeywordTok{solve}\NormalTok{(A))}
\end{Highlighting}
\end{Shaded}

\begin{verbatim}
## [1] 0.04
\end{verbatim}

\begin{Shaded}
\begin{Highlighting}[]
\CommentTok{# det(A)−1 = 1 / det(A)}
\DecValTok{1} \OperatorTok{/}\StringTok{ }\KeywordTok{det}\NormalTok{(A)}
\end{Highlighting}
\end{Shaded}

\begin{verbatim}
## [1] 0.04
\end{verbatim}

\begin{Shaded}
\begin{Highlighting}[]
\KeywordTok{det}\NormalTok{(A }\OperatorTok\StringTok{ }\NormalTok{B)}
\end{Highlighting}
\end{Shaded}

\begin{verbatim}
## [1] 100
\end{verbatim}

\begin{Shaded}
\begin{Highlighting}[]
\KeywordTok{det}\NormalTok{(A) }\OperatorTok{*}\StringTok{ }\KeywordTok{det}\NormalTok{(B)}
\end{Highlighting}
\end{Shaded}

\begin{verbatim}
## [1] 100
\end{verbatim}

\begin{Shaded}
\begin{Highlighting}[]
\CommentTok{#3.}
\KeywordTok{t}\NormalTok{(}\KeywordTok{solve}\NormalTok{(A))}
\end{Highlighting}
\end{Shaded}

\begin{verbatim}
##       [,1]  [,2]  [,3]  [,4]
## [1,]  0.08 -0.24 -0.12 -0.52
## [2,]  0.48 -0.44  0.28 -1.12
## [3,] -0.44  0.32  0.16  0.36
## [4,]  0.20 -0.60  0.20 -0.80
\end{verbatim}

\begin{Shaded}
\begin{Highlighting}[]
\KeywordTok{solve}\NormalTok{(}\KeywordTok{t}\NormalTok{(A))}
\end{Highlighting}
\end{Shaded}

\begin{verbatim}
##       [,1]  [,2]  [,3]  [,4]
## [1,]  0.08 -0.24 -0.12 -0.52
## [2,]  0.48 -0.44  0.28 -1.12
## [3,] -0.44  0.32  0.16  0.36
## [4,]  0.20 -0.60  0.20 -0.80
\end{verbatim}

\begin{Shaded}
\begin{Highlighting}[]
\KeywordTok{t}\NormalTok{(A }\OperatorTok\StringTok{ }\NormalTok{B)}
\end{Highlighting}
\end{Shaded}

\begin{verbatim}
##      [,1] [,2] [,3] [,4]
## [1,]    6    5   -3  -11
## [2,]    3   -1    4    5
## [3,]  -13   13    0  -11
## [4,]    8  -14    3   17
\end{verbatim}

\begin{Shaded}
\begin{Highlighting}[]
\KeywordTok{t}\NormalTok{(B) }\OperatorTok\StringTok{ }\KeywordTok{t}\NormalTok{(A)}
\end{Highlighting}
\end{Shaded}

\begin{verbatim}
##      [,1] [,2] [,3] [,4]
## [1,]    6    5   -3  -11
## [2,]    3   -1    4    5
## [3,]  -13   13    0  -11
## [4,]    8  -14    3   17
\end{verbatim}

\begin{Shaded}
\begin{Highlighting}[]
\KeywordTok{solve}\NormalTok{(A }\OperatorTok\StringTok{ }\NormalTok{B)}
\end{Highlighting}
\end{Shaded}

\begin{verbatim}
##       [,1]  [,2]  [,3]  [,4]
## [1,] -1.06 -4.36  4.58 -3.90
## [2,]  0.63  2.28 -2.09  1.95
## [3,] -1.59 -6.04  6.37 -5.35
## [4,] -1.90 -7.40  7.70 -6.50
\end{verbatim}

\begin{Shaded}
\begin{Highlighting}[]
\KeywordTok{solve}\NormalTok{(B) }\OperatorTok\StringTok{ }\KeywordTok{solve}\NormalTok{(A)}
\end{Highlighting}
\end{Shaded}

\begin{verbatim}
##       [,1]  [,2]  [,3]  [,4]
## [1,] -1.06 -4.36  4.58 -3.90
## [2,]  0.63  2.28 -2.09  1.95
## [3,] -1.59 -6.04  6.37 -5.35
## [4,] -1.90 -7.40  7.70 -6.50
\end{verbatim}

\end{document}
